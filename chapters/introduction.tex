\Chapter{Bevezetés}

%A fejezet célja, hogy a feladatkiírásnál kicsit részletesebben bemutassa, hogy miről fog szólni a dolgozat.
%Érdemes azt részletezni benne, hogy milyen aktuális, érdekes és nehéz probléma megoldására vállalkozik a dolgozat.

%Ez egy egy-két oldalas leírás.
%Nem kellenek bele külön szakaszok (section-ök).
%Az irodalmi háttérbe, a probléma részleteibe csak a következő fejezetben kell belemenni.
%Itt az olvasó kedvét kell meghozni a dolgozat többi részéhez.
A BPEL (\textit{Business Process Execution Language}) nyelv létrejötte elsődlegesen a Web szolgáltatások területéhez kapcsolódik, de a nyelv mint általános munkafolyamat (\textit{workflow}) leíró nyelv, más alkalmazási témakörökhöz is köthető. A BPEL szerepét, fontosságát jól mutatja az a tény is, hogy igen gazdag irodalom található az egyes alkalmazási területekről és speciális szabvány kiegészítésekről.
%TODO Ide kellenének a hivatkozások néhány irodalomról!

A  BPEL aktualitását jelzi, hogy a megvalósító motorok köre is folyamatosan bővül. Ugyan már lassan 15 év eltelt a szabvány bevezetése óta, a meglévő nagyobb rendszerek (Oracle BPEL Process Manager, IBM WebSphere Process Server, Microsoft BizTalk Server, SAP SAP Exchange Infrastructure) alternatívájaként  most is jelennek meg új végrehajtó motor implementációk. A Wikipédia forrása szerint \cite{wikiBpelList} a közelmúltban az alábbi szabad szoftver implementációk születtek (\ref{tab:bpel_softwares}. táblázat).

\begin{table}[h!]
\centering
\caption{A BPEL nyelv szabad szoftveres implementációi.}
\label{tab:bpel_softwares}
\begin{tabular}{|c|c|c|c|}
\hline
\textbf{Termék neve} & \textbf{Fejlesztő} & \textbf{Megjelenés éve} & \textbf{Licensz}\\
\hline
JBPM & JBoss & 2016 & Apache\\
Apache ODE & ASF & 2016 & Apache\\
Activiti & Alfresco & 2014 & Apache\\
\hline
\end{tabular}
\end{table}

Annak ellenére, hogy napjainkra már több BPEL motor elérhető és használatos, a BPEL szerkesztők és különösen a BPEL validációs rendszerek köre igen szegényes. Ezen tapasztalatokból kiindulva a dolgozat célja egy olyan BPEL validációs rendszer elkészítése, amely a BPEL rendszerek egyik fontos tulajdonságát, a terhelés korlátosságát (\textit{bounded model})

vizsgálja. A korlátosság azt jelzi, hogy minden csomópontban van egy felső korlát a végrehajtható feladatok számára, intenzitására vonatkozóan. Ha a rendszer nem teljesíti ezt a kritériumot, akkor túlcsordul valamely megmunkáló/tároló helyen. Az elemzés során a korlátosság ténye mellett, a korlát értékei is fontos vizsgálandó jellemzők. 

A meglévő tervezői rendszerekben legtöbbször szimulációval történik a főbb paraméterek, a korlátosság vizsgálata. Ezen megközelítésnek rendszerint két problémája van: a vizsgálat teljessége (azaz valóban minden lehetséges esetet áttekintettük-e) illetve a végrehajtási idő (a szimulációk futtatása hosszabb időt is igénybe vehet).
\newpage
A dolgozatban a BPEL folyamatok Petri-háló alapú vizsgálatát végzem el. A Petri-háló alapú reprezentáció egy elfogadott és többek által alkalmazott megközelítés. A kidolgozott rendszer inputként egy  BPEL modell leírását várja és kimenetként az elemzés eredményét illetve a folyamatok nyomkövetését adja vissza. A dolgozatban először bemutatásra kerül a bemenet tehát a BPEL dokumentum, majd a végtermék a Petri-háló és elemzése, majd az, miképp lehet az outputot előállítani.