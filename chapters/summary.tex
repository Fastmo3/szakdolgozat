\Chapter{Összefoglalás}

A feladatot természetesen a BPEL megismerésével kezdtem. Majd miután egy alapvető képem kialakult a rendszerről, megpróbáltam saját magam is BPEL file létrehozását. Különböző próbálkozások után kialakult egy átfogó kép a folyamatokról, illetve a BPEL használhatóságáról. A következő lépés a Petri hálók megismerése volt. A programhoz szükség volt különböző modellekre, melyek a hálót írják le. A modelleket úgy kellett létrehozni, hogy rajtuk egyszerű legyen a tulajdonság vizsgálat, de ennek ellenére leírható legyen egy egész folyamat a teljesség igényével. 
A hálók vizsgálatához vissza kellett térni a diszkrét matematika, a matematikai logika és automaták és formális nyelvek tárgyból tanultakra. 

Végeredményképp a program képes egy (pseudo) BPEL file feldolgozására. Az inputot konvertálja Petri hálóra a beállítások alapján, és az így kapott hálón elemzéseket hajt végre, és közli, hogy a háló véges-e, korlátos-e, illetve, van-e szükség belső puffer használatára. Az elkészült hálók a programból megtekinthetők, és rajtuk a tokenáram lépésről lépésre megvizsgálható. A hálómegjelenítés egyedüli korlátozója a képernyőméret, illetve a felbontás.  A színes háló előállításakor, mivel nem egyértelmű a színezés ezért lehet többféle hálót is létrehozni. A program azonban csak egyet állít elő. Ezt azonban a definíciós szabálykészlet bővítésével könnyen lehet módosítani. Az önkényesen megválasztott színezés miatt egy, akár színes hálóra konvertálható folyamatot hamisan, konvertálhatatlanként kezel, de ez ha szükség adódna rá javítható több színezési algoritmus implementálásával. A fentebb említett gráfmegjelenítés szintén javítható egy olyan gráfrajzoló használatával, ami támogatja a vektoros formátumot, illetve a kapott kép átméretezését, vagy a részkép illesztéses megjelenítését. 